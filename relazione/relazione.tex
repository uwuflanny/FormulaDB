\documentclass[a4paper,12pt]{report}
\usepackage[italian]{babel}
\usepackage[italian]{cleveref}

\usepackage{booktabs} % For prettier tables
\usepackage{graphicx}
\usepackage{array}

\title{Relazione\break``progetto FormulaDB''}

\author{Migliarini Gianluca - Montali Giacomo}
\date{Giugno 2021}

\begin{document}
	
\maketitle
\chapter{Analisi dei requisiti}
	\section{intervista}
		Si vuole tenere traccia dei campionati del mondo Formula, memorizzando per ciascuno l'anno del campionato e la categoria di auto che corre al suo interno.
		
		In ogni campionato gareggiano circa 20 piloti, dei quale vengono salvate informazioni quali il ed il cognome del concorrente,
		la nazionalita', la data di nascita ed il numero di macchina con il quale corre.
		
		Per poter gareggiare, ogni pilota stipula un contratto con una scuderia, la quale gli offre un veicolo, con il quale
		prendere parte alle competizioni. Il contratto ha solitamente durata di qualche anno, tuttavia, in rare occasioni,
		la scuderia concede al pilota di gareggiare per un team diverso.
		
		Per ogni scuderia si tiene traccia del suo nome e la nazione per la quale corre.
		
		Nei test effettuati durante il periodo di pausa tra le varie competizioni, eseguiti da ingegneri specializzati appartenenti al team,
		permettono alla scuderia di migliorare la propria autovettura, offrendo cosi' la possibilita' di gareggiare
		con un nuovo modello per il campionato che verra'. In particolare, le migliorie apportate interessano il peso e le dimensioni dell'auto.
		Inoltre, le scuderie, in caso di budget ridotto, possono eventulmente acquistare il motore da team avversari.
		
		Per ogni scuderia, e' necessario tenere traccia degli ingegneri che ci lavorano e della loro specializzazione.
		
		Di ogni Gran Premio viene memorizzato, oltre alla data, al numero di giri ed il meteo;
		la posizione ed il nome del circuito dove viene disputato.
		
		Per quanto riguarda i circuiti, e' necessario permetterne la localizzazione salvando il nome, la nazione in cui sono situati e il loro indirizzo.
		
		Inoltre, per una memorizzazione migliore di ogni gara, viene memorizzato ogni giro di ogni pilota effettuato in gara con il rispettivo tempo,
		i pitstop effettuati con il tempo impiegato in essi, l'ordine di partenza (dato dalle qualifiche) e l'ordine di arrivo.
		
	\hfill
	\small\addtolength{\tabcolsep}{+15pt}
	\begin{table}[h!]
		\begin{center}
			\caption{Stima volume dati}
			\label{tab:table1}
			
			\scalebox{1.5}{
			\begin{tabular}{l|c|r} % <-- Alignments: 1st column left, 2nd middle and 3rd right, with vertical lines in between
				\toprule
				\textbf{Dato} & \textbf{Tipo} & \textbf{Quantità}\\
				
				\midrule		
				Nazione				&	E	&	210\\     
				Circuito			&	E 	&	100\\
				scuderia			&	E	&	35\\
				Pilota				&	E	&	70\\
				contratto\_pilota	&   R 	&	100\\
				Contratto			&	R 	&	1150\\
				ingegnere			&	E 	&	1000\\
				Campionato			&	E	&	15\\	     
				Giro 				&	E 	&	18000\\
				info\_gara			&	E   &	300\\ 
				riepilogo			&	R	&	6000\\
				risultati\_gara		&	E	&	6000\\
				risultati\_qualifica&	E	&	6000\\
				pit\_stop			&	E	&	7500\\
				Motore				&	E	&	18\\
				Macchina			&	E	&	175\\
				
				\bottomrule
			\end{tabular}}
		\end{center}
	\end{table}
	
	\section{Operazioni principali e frequenza:}
	Le operazioni da effettuare sono quelle già elencate nella fase di analisi. Segue una tabella
	riportante la loro descrizione e relativa frequenza:
	
	\hfill
	\small\addtolength{\tabcolsep}{+15pt}
	\begin{table}[h!]
		\begin{center}
			\caption{Stima volume dati}
			\label{tab:table1}
			
			\scalebox{1.1}{
				\begin{tabular}{l|r} % <-- Alignments: 1st column left, 2nd middle and 3rd right, with vertical lines in between
					\toprule
					\textbf{Operazione} & \textbf{Frequenza}\\
					\midrule
					aggiungere un pilota								& 4 / anno\\
					aggiungere una scuderia								& 1 / 3 anni\\
					aggiungere un motore								& 4 / 3 anni\\
					aggiungere macchina ad una scuderia					& 10 / anno\\
					aggiungere un circuito								& 1 / 5 anni\\
					aggiungere un campionato							& 1 / anno\\
					aggiungere gara ad un campionato					& 20 / anno\\
					aggiungere un riepilogo di un pilota				& 1200 / anno\\		
					ottenere la classifica piloti di un campionato  	& 20 / anno\\
					ottenere la classifica scuderie di un campionato 	& 20 / anno\\
					ottenere numero vittorie di ogni pilota				& 12 / anno\\
					ottenere classifica veicoli di una certa categoria	& 10 / anno\\
					\bottomrule
			\end{tabular}}
		\end{center}
	\end{table}
	
	
\end{document}